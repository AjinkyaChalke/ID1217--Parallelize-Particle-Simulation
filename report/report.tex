\documentclass[titlepage,a4paper,10pt]{article}
% Språk och encodings
\usepackage[swedish,english]{babel}
\usepackage[T1]{fontenc}
\usepackage[utf8]{inputenc}
\usepackage[fixlanguage]{babelbib}
% Images and floats
\usepackage{graphicx}
\usepackage{wrapfig}
\usepackage{float}
% Clear type + Sans-serif font
\usepackage{lmodern}
\renewcommand{\familydefault}{\sfdefault}
% Matte
\usepackage{amsmath, amsthm, amssymb}
% Algoritmer
\usepackage[ruled,vlined]{algorithm2e}
% Länkar
\usepackage{color}
\definecolor{dark-blue}{rgb}{0, 0, 0.6}
\usepackage{hyperref}
\hypersetup{
  colorlinks=true,
  linkcolor=dark-blue,
  urlcolor=dark-blue
}
% Vettiga paragrafer
\setlength{\parindent}{0pt}
\setlength{\parskip}{2ex}

% Sidhuvud/sidfot
\usepackage{fancyhdr}
\setlength{\headheight}{15pt}
\pagestyle{fancyplain}
\lfoot{Carl-Oscar Erneholm \\ 880422-0872 \\ coer@kth.se}
\rfoot{Martin Nycander \\ 881028-0076 \\ mnyc@kth.se}
\cfoot{Page \thepage}

% Språk
\selectbiblanguage{swedish}
\selectlanguage{swedish}

% Titel
\title{ID1217: Parallel Particle Simulation}
\author{Martin Nycander \and Carl-Oscar Erneholm}
\date{\today}

\begin{document}
\maketitle

\tableofcontents
\newpage

\setcounter{page}{1}
% In your report you should explain what you have done and what you have learned. 
% Your report should be few (about 10) pages of text plus tables and figures and an optional Appendix. 

\section{Problem definition}
\begin{equation}
    gridSize = \frac{\sqrt{0.0005 \cdot n}}{0.01} + 1 \\
\end{equation}
\begin{equation}
    grid.size = gridSize^2 = \left(\frac{\sqrt{0.0005 \cdot n}}{0.01} + 1\right)^2
\end{equation}

\section{Problem solutions}

% TODO: A description of the synchronization you used in the shared memory implementation.

% TODO: A description of the communication you used in the distributed memory implementation.

% TODO: A description of the design choices that you tried and how did they affect the performance.

\section{Results}

% TODO: A plot in log-log scale that shows that your serial and parallel codes run in O(n) time and a description of the data structures that you used to achieve it. In order to get more precise timing estimates, we recommend you to run a program at least 5 times and take the median (rather than the mean) of the simulation times.

% TODO: Speedup plots that show how closely your parallel codes approach the idealized p-times speedup and a discussion on whether it is possible to do better.

% TODO: Where does the time go? Consider breaking down the runtime into computation time, synchronization time and/or communication time. How do they scale with p?

\section{Discussion}

% TODO: A discussion on using pthreads, OpenMP and MPI.

\end{document}
