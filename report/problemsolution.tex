
%TODO: describe how we made the serial program O(n), talk about grids.
    \subsection{Simulation Design}

        Every frame in the original implementation made $n^2$ computations. For
        each particle it calculated the distance to each other particle, if the
        particle was within the cutoff range it would also calculate and apply
        the appropriate force to the particle. The problem with this is that the
        algorithm checks the distance to all other particles when clearly not
        all of them are relevant. The solution we used was to divide the plane
        into a grid of squares, we then mapped each particle to a cell in the
        grid. By making sure that the sides of the cells is not less than the
        cutoff range, we can be sure that all relevant particles to a given
        particle are within the Moore neighbourhood of the cell. This means that
        for each cell we need to check 8 other cells, which is constant work.

        And as can be shown by the formulas below the total grid size has a
        linear relation with $n$.

        \begin{equation*}
            \text{grid edge length} = \frac{\sqrt{0.0005 \cdot n}}{0.01} + 1 \\
        \end{equation*}
        \begin{equation*}
            \text{grid total size} = \text{grid edge length}^2 = \left(\frac{\sqrt{0.0005 \cdot n}}{0.01} + 1\right)^2 \approx O(n)
        \end{equation*}

        As such, our improved algorithm should run in linear time.

    \subsection{Shared Memory Solution Design}

    Making the algorithm parallel with the shared memory API's are pretty
    simple. We give each processor a distinct set of particles to keep track
    of. With such a setup we can run the original solution practically
    unmodified with the exception that we must identify the critical sections
    and run them in mutual exclusion.

    The parts which have critical sections are the ones writing to the grid
    structure and in relation to the entire code it's not so bad.

    \subsection{Message Passing Solution Design}

    Converting a shared memory solution to a message passing solution is quite a
    big step. Here we have to redesign a bit to have minimum amount of memory
    sharing.

    What we did was that each thread got assigned a number of rows of the grid
    matrix instead of a partion of the particles array. This way, each thread
    was responsible for a set of particles which were geografically close.
    Functionality was then implemented for transferring particles between
    threads and also sending over the border-rows since we still need the
    Moore-neighbourhood of each particle.
